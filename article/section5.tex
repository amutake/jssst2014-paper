
\section{計算量に関する結果と考察}

効率の二つの尺度のうち,スループットにつ
いては容易に議論ができる.すなわち,二つの操作は,レベル$l$ (根をレ
ベル$0$として)の節点を${\rm O}(l)$回 --- ${\it update\/}$は高々$(l+2)$回,
zippingは高々$(2l+2)$回 ---
の回転操作ののちに確定させる.
さらにどちらの操作も,連続する高々$3$
レベルの節点を同時に施錠するだけでよい.これらのことから,木の大きさや深
さによらないスループットで,操作系列をパイプライン的に並列処理するこ
とができる.

レスポンスは,${\it update\/}$については,通常のスプレー
木と同等の償却計算量をもつことが証明できる.具体的には,
節点$x$の{\bf 大きさ}$s(x)$を$x$を根とする部分木の節点数と定義し,
{\bf ランク}$r(x)$を$\log_2(s(x))$とする.
そして
%
木の{\bf ポテンシャル}を,すべての節点のランクの和と定義する.
すると,${\it update\/}$の償却時間,つまり回転操作の回数で測った所
要時間に操作前後のポテンシャルの変化を加えたものは,$n$を木の節点数とし
て,${\rm O}(\log n)$であることを示すことができる.
このことから,十分長い操作系列の平均レスポンスは,最悪でも対数的であるこ
とがわかる.
文献\Cite{ST85}のよう
に,節点に異なる重みをつけて$s$や$r$を定義することにより,より強い性質
を示すこともできるが,本論文では省く.

一方,${\it delete\/}$については,文献\Cite{ST85}の解析方法では,対
数的償却計
算量を導くことはできない.そのことを示すために,図\ref{figure:delete}
(b)の4回の回転によるポテンシャル変化を考える.

図\ref{figure:delete}(b)の一番右側
の木のランク関数を$r'$とする.一番左側の木からのポテンシャルの変化を,
$k$をある正定数として$k(r'(b)-r'(z))$以内に押さえることができることを示すのが,
文献\Cite{ST85}における償却計算量の証明技法の基本であった.しかし,
これらの木に
ついて$s(A)= s(B) = s(C) = h\gg t = s(D) = s(E) = s(F)$
を仮定すると,ポテンシャル変化が$h/t$に関して${\rm O}(\log
(h/t))$となる.一方$r'(b)-r'(z)$は$h/t$に関して${\rm O}(1)$であるので,
上記の要請を満たす
$k$は存在しないことがわかる.Zippingに先立ってパス短縮化を行なっ
た場合についても,同様のことが示せる.

しかし,第\ref{section:delete}節の削除操作は,
%
アクセスしたパス上の節点の深さが約半分になり(事前にパス
短縮化を施した場合),それ以外の節点も高々定数レベルしか沈まない
%
という,節点の浮き沈みについてのスプレー木一般の性質は満たしている.
%
では一般に,この二つの性質を満たす自己調整的な木アルゴリズムで,平均レス
ポンスが対数時間で押さえられないような,十分長い操作系列は存在す
るのだろうか? これは未解決であるが,本論文で提案した二操作に
ついては,平均レスポンスは少なくとも${\rm O}(\sqrt n)$ (更新のみならば
${\rm O}(\log n)$)と予想される.

その根拠
として,各節点の削除しやすさの変化を考える.
節点$x$の{\bf 削除困難度}$d(x)$を,$x$からその直前のキー$x_-$をもつ節
点へ至るパス長($x_-$が存在しない場合や,$x_-$が$x$の子孫で
ない場合は$0$と定める)と直後のキー$x_+$をもつ節に至るパス長の最小値
と定めると,第\ref{section:delete}節の
${\it delete\/}$は,$d$の大きな節点の消去には時間がかかるも
のの,残った各節点の$d$を高々${\rm O}(1)$
しか大きくしない.また第\ref{section:update}節の
${\it update\/}$で新たに挿入した節点の$d$
は$0$であり,${\it update\/}$はすでに存在していた各節点の$d$も高々${\rm
O}(1)$しか大きくしない.(ボトムアップ扁平化における節点の$d$の増加は,定数で
押えることができない.)これらのことから
%
\begin{enumerate}
\item[1.]
新たな節点の$d$の値が$k$まで成長するには,他の節点の$\Omega(k)$
回の挿入削除が必要
\end{enumerate}
%
であることがわかる.さらに
%
\begin{enumerate}
\item[2.]
二分木における各節点の$d$の総和は,
木をトラバースしたときに通る枝の延べ本数を上回ることはないから
${\rm O}(n)$
\end{enumerate}
%
である.1.と2.から,
新たな節点の挿入と,$d$の大きな節点の消去が繰り返されるという最悪の操作
系列を考えても,操作の平均の手間は${\rm O}(\sqrt n)$であり,実用上の効率
は更新操作のみの場合とほとんど変わらないと予想される.
